\documentclass{article}
\usepackage[utf8]{inputenc}
\usepackage[T1]{fontenc}              
\usepackage[french]{babel}
\usepackage[a4paper, margin=3cm]{geometry}
\usepackage{amsmath, amssymb, graphicx, xcolor, hyperref}


\title{Global Minimum for Active Contour Models: A Minimal Path Approach}
\author{JACOB Amaury}
\date{\today}

\begin{document}

\maketitle
\tableofcontents

\section{Introduction}
Le but de ce devoir est de comprendre, reproduire et critiquer un article.
Mon article porte sur les contours actifs par une approche du chemin minimal.
La méthode présentée dans cet article est une amélioration de la méthode des Snakes ;
cependant, cette méthode ne nécessite que deux points comme initialisation.
Elle a aussi pour avantage d'être moins sensible aux minima locaux
et de donner le chemin minimal entre les deux points.

Dans un premier temps, nous allons voir les bases mathématiques nécessaires
pour comprendre l'article.

\section{Comment calculer un contour avec un chemin minimal}
\subsection{Calcule du potentiel $P$}

Pour calculer le contour à l’aide du chemin minimal, il faut d’abord construire une carte de potentiel.  
Cette carte repose sur le gradient de l'image et sur la norme du gradient.

\begin{equation}
\mathbf{\nabla I} =
\begin{bmatrix}
I'_x \\[4pt]
I'_y
\end{bmatrix}
\quad\text{et}\quad
\|\nabla I\| = \sqrt{(I'_x)^2 + (I'_y)^2}.
\end{equation}

Maintenant que nous avons calculé la norme du gradient, nous pouvons déterminer le potentiel $P$.
Le potentiel $P(x,y)$ agit comme une mesure de "résistance" au passage du contour.
Ainsi, plus le gradient est fort (bord marqué), plus le potentiel est faible.
Le chemin minimal cherchera donc naturellement à suivre les bords de l’image.


Dans l’article de référence, aucune expression explicite n’est donnée pour $P$.  
J’ai donc cherché dans d’autres sources des définitions possibles du potentiel, en particulier dans deux travaux
où les formules sont simples et peu coûteuses à calculer d’un point de vue numérique :

\begin{equation}
\begin{aligned}
P_1(x,y) &= e^{-\alpha \|\nabla I(x,y)\|^2}, \\
P_2(x,y) &= \frac{1}{1 + \alpha \|\nabla I(x,y)\|^2}.
\end{aligned}
\end{equation}

Ces deux définitions apparaissent respectivement dans les articles de  
Caselles \textit{et al.}, \textit{“Geodesic Active Contours”} (1997), et  
Sethian \& Kimmel, \textit{“Computing Geodesic Paths”} (1998).

C'est deux expression sont similaire mais une décroit de facons exponentiel 
alors que l'autre décroit en $\frac{1}{x}$ ce qui est plus lent. Le choix va
dépendre de l'image. 

\subsection{Calcule de la métrique $\tilde{P}$}
Une fois la carte de potentiel $P(x,y)$ calculée, elle est utilisée pour définir 
une métrique de longueur $\tilde{P}$. 
\begin{equation}
\tilde{P} = \omega + P(p)
\end{equation}
$\omega$ est un terme qui représente le lissage de la courbe et $p$ est le point actuel.
Cette métrique joue le role de vitesse dans l'équation Eikonal que nous allons voir. 


\subsection{Equation Eikonal}
L'équation Eikonal relie $\tilde{P}(x,y)$ et une fonction de coût $U(x,y)$. Cette fonction
représente en chaque point de l'image le coût pour aller du point de départ a un point $(x,y)$
de l'image.
\begin{equation}
\|\nabla U(x,y)\| = \tilde{P}(x,y),
\end{equation}
avec la condition initiale $U(p_0)=0$.

Dans l’article de Cohen et Kimmel, cette équation est résolue numériquement
par la méthode du \textit{Fast Marching}.
Cette méthode repose sur une propagation du front de niveau de $U$ :
le front avance plus rapidement dans les zones où $\tilde{P}$ est faible (autour des bords)
et plus lentement dans les zones où $\tilde{P}$ est élevée.

Mathématiquement, l’espace est discrétisé en une grille, et pour chaque pixel $(i,j)$,
on cherche la valeur de $U_{i,j}$ qui satisfait la version discrète de l’équation :
\begin{equation}
(\max\{U - U_{i-1,j}, U - U_{i+1,j}, 0\})^2 + 
(\max\{U - U_{i,j-1}, U - U_{i,j+1}, 0\})^2 = \tilde{P}_{i,j}^2.
\end{equation}
Cette équation quadratique est résolue localement à chaque itération
en utilisant uniquement les voisins déjà mis à jour (schéma \textit{upwind}),
ce qui garantit la cohérence du front de propagation.

\subsection{Back-propagation du chemin minimal}





\end{document}
